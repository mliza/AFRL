\input{/Users/Martin/Documents/Documents/gitRepo/Scripts/Latex/templates/homeworkTemplate}
\input{/Users/Martin/Documents/Documents/gitRepo/Scripts/Latex/macros/equationMacros}
\input{/Users/Martin/Documents/Documents/gitRepo/Scripts/Latex/macros/mathMacros}
\input{/Users/Martin/Documents/Documents/gitRepo/Scripts/Latex/macros/pictureMacros}
\Header{\today}{Script's Explanation}{Martin E. Liza}

\begin{document}    
    \section*{Variables:}
    \begin{itemize}
        \item $ \Sub{\epsilon}{0}:\text{ vacuum permittivity}\UnitsFrac{A^2S^4}{m^3kg} $ 
        \item $ \Sub{N}{A}:\text{ Avogadro's number}\UnitsFrac{1}{mol} $
        \item $ \Sub{M}{i}:\text{ specie's molar mass}\UnitsFrac{kg}{mol} $
        \item $ \Sub{A}{i}: \text{ specie's molar refractivity in GS units}\UnitsFrac{cm^3}{mol} $ 
        \item $ \UpSub{\alpha}{i}{vol}:\text{ species's polarizability volume in CGS units}\Units{cm^3} $
        \item $ \Sub{\alpha}{i}:\text{ specie's polarizabiliy in SI units}\UnitsFrac{A^2s^4}{kg} $
        \item $ \Sub{R}{GD}:\text{ Gladstone-Dale constant}\UnitsFrac{m^3}{kg} $
        \item $ \Up{\alpha}{spec}:\text{ specific polarizability}\UnitsFrac{A^2s^4}{m^3kg^2} $
        \item $ n:\text{ index of refraction}\Units{\;} $
        \item $ \Sub{\rho}{i}:\text{ specie's density}\UnitsFrac{kg}{m^3} $
        \item $ \Sub{\rho}{medium}:\text{ medium's density}\UnitsFrac{kg}{m^3} $
    \end{itemize}
        
    \section{Calculating the Gladstone-Dale constant $\Parenthesis{\Sub{R}{GD}}$ from the polarizability volume $\Parenthesis{\Up{\alpha}{vol}}$:}
    \indent From \href{https://en.wikipedia.org/wiki/Polarizability}{wikiPolarizability}, it was found the conversion factor from the polarizability volume in CGS units to polarizability ins SI units, to be given by equation \eqref{cgsToSI}. 
    \begin{equation}\label{cgsToSI}
        \alpha = 4\pi\Sub{\epsilon}{0}\Up{\alpha}{vol}\times10^{-6}\UnitsFrac{A^2s^4}{kg} 
    \end{equation}
    \indent The molar refractivity is given by equation \eqref{molarRefractivity} 
    \begin{equation}\label{molarRefractivity}
        A = \frac{4\pi}{3}\Sub{N}{A}\Up{\alpha}{vol}\UnitsFrac{cm^3}{mol} 
    \end{equation}
    \indent The molar refractivity $\Parenthesis{A}$ in CGS units, can be converted to SI units, using equation \eqref{cgsToSI}
    $$ \therefore \Up{A}{SI} = 4\pi\Sub{\epsilon}{0}A\times 10^{-6}\UnitsFrac{A^2s^4}{kg\,mol} $$
    \indent The specific polarizability is given by \eqref{specPolarizability}
    \begin{equation}\label{specPolarizability}
        \UpSub{\alpha}{i}{spec} = \frac{\Up{A}{SI}}{\Sub{M}{i}}\UnitsFrac{A^2s^4}{kg^2}
    \end{equation}
    \indent The Gladstone-Dale for each specie can be calculated by equation \eqref{specGladstoneDale}
    \begin{equation}\label{specGladstoneDale}
        \Sub{R}{GD,i} = \frac{\UpSub{\alpha}{i}{spec}}{2\Sub{\epsilon}{0}}\UnitsFrac{m^3}{kg}
    \end{equation}
    \indent The Gladstone-Dale constant for the medium is calculated by equation \eqref{gladstoneDaleMedium}
    \begin{equation}\label{gladstoneDaleMedium}
        \Sub{R}{GD} = \sum_{i}^{\infty}\frac{\Sub{R}{GD,i}}{\Sub{\rho}{medium}}\Sub{\rho}{i}\UnitsFrac{m^3}{kg} 
    \end{equation}
    \indent The index of refraction can be calculated by the Gladstone-Dale relation given on \eqref{indxOfRefraction}
    \begin{equation}\label{indxOfRefraction}
        n = 1 + \Sub{R}{GD}\Sub{\rho}{medium}\Units{\;}
    \end{equation}
    \indent The OPL is calculated by equation \eqref{eqOPL}
    \begin{equation}\label{eqOPL}
        OPL = \int n \,dl 
    \end{equation}
    \indent Lastly, the OPD is calculated by equation \eqref{eqOPD}
    \begin{equation}\label{eqOPD}
        OPD = OPL - \Avg{OPL}
    \end{equation}

\end{document}
