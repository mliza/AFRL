
\documentclass[11pt]{article}
\usepackage[top=1in, bottom=1in, right=0.68in, left=0.68in]{geometry} %paper size 
\usepackage{sectsty} %unbolt section and modifies section titles 
\usepackage{indentfirst} %indent first line in every paragraph  
\usepackage{amssymb}%load numerate letters
\usepackage{enumitem}%load math notations 
\usepackage{amsfonts} %load more math fonts
\usepackage{graphicx} %load graph
\usepackage[usenames,dvipsnames]{color} %load color 
\usepackage{cancel}% cancel a term 
\usepackage{amsmath}%box an equation 
\usepackage{bm} %bold everything also Greek letter
\usepackage{lipsum,lmodern} %generates text filler 
\allsectionsfont{\large\mdseries} %modify sections/subsections in the document 
\usepackage{float} %force a figure to stay in a location 
\usepackage{caption}  %figure captions
\usepackage{subcaption} %figure sub captions 
\usepackage{sidecap} %SC figure 
\usepackage{tikz} %lets you draw 
\usepackage[most]{tcolorbox} %color box
\usepackage{color,soul}%highlight 
\usepackage{scrlayer-scrpage} %allows to control page footages and headers  
\usepackage{lastpage} %footage
\usepackage{blindtext} %headers
\usepackage[toc,page]{appendix} %appendix
\usepackage[colorlinks]{hyperref} %hyper link references
\usepackage{esint,relsize} %mathlarge
\usepackage{comment} %comment out
\usepackage{pdfpages} %insert PDF files 
\usepackage{footmisc} %allows footnote references 
%\usepackage{draftwatermark} 	%Draft mark package
\restylefloat{table} %float table 
\graphicspath{{figures/}} %Setting the graphicspath 
%\usepackage[backend=biber]{bibliatex} %only use if you are not building a bibiliography

%Headers/footers   
\clearscrheadfoot
\newcommand{\Header}[3]
{  
\ihead{#1} 	%left 
\chead{#2}  %center
\ohead{#3}  %right
 }
\ofoot{Page \textbf{\thepage} of \textbf{\pageref{LastPage}}} %page numbers 

%Modify clickable links color 
\hypersetup{ colorlinks={true},           %active color links 
             urlcolor={blue},             %url link color 
             linkcolor={black},            %change link color
             citecolor={blue}   
           }
            

%New Commands Text
\newcommand{\MyName}{My Name}   %my name
\newcommand{\DocumentName}{Document Name} %document name

%Rename Commands Text
\renewcommand\thesubsection{(\alph{subsection})} 	%subsection 
\renewcommand\thesection{\textbf{\arabic{section})}} 		%section
\renewcommand{\theenumi}{(\roman{enumi})} 			%enumerate sections
\renewcommand{\labelenumi}{\theenumi} 				%enumerate subsections

%Equations for this document
%\def\name{"equation goes in here"} %define eq that will be use several times on document, call$\name$

%Rename math commands
\everymath{\displaystyle} %display style for everymath 
\newcommand{\B}{\left}	  
\newcommand{\E}{\right}
\newcommand{\Acronym}[2]{\B( #1_{_{#2}}\E)}
\newcommand{\Avg}[1]{\langle#1\rangle}
\newcommand{\Parenthesis}[1]{\B(#1\E)}
\newcommand{\SqBracket}[1]{\B[#1\E]}
\newcommand{\Sub}[2]{#1_{_{#2}}} 
\newcommand{\Up}[2]{#1^{^{#2}}} 
\newcommand{\UpSub}[3]{#1_{_{#2}}^{^{#3}}}
\newcommand{\Result}[1]{\bm{\boxed{#1}}}
\newcommand{\Limit}[2]{\underset{#1\rightarrow#2}{\lim}} 
\newcommand{\Function}[2]{#1_{\Parenthesis{#2}}} 
\newcommand{\Units}[1]{\,\SqBracket{#1}} 
\newcommand{\UnitsFrac}[2]{\,\SqBracket{\frac{#1}{#2}}} 
\newcommand{\Abs}[1]{\B\lvert#1\E\rvert} 
\newcommand{\Norm}[1]{\B\lVert#1\E\rVert} 

% Derivatives 
\newcommand{\Partial}[1]{\Sub{\partial}{#1}} 
\newcommand{\PartialFrac}[2]{\frac{\partial #1}{\partial #2}} 
\newcommand{\PartialFracN}[3]{\frac {\Up{\partial}{#3} #1}{\partial \Up{#2}{#3}}} 
\newcommand{\PartialDer}[2]{\frac{\partial #1}{\partial #2}} 
\newcommand{\PartialDerN}[3]{\frac {\Up{\partial}{#3} #1}{\partial \Up{#2}{#3}}} 
\newcommand{\TotalDer}[2]{\frac{d #1}{d #2}} 

% Trig  
\newcommand{\Sin}[1]{\sin\Parenthesis{#1}}
\newcommand{\SinFrac}[3]{\sin\Parenthesis{ \frac{#1}{#3} #2}}
\newcommand{\Cos}[1]{\cos\Parenthesis{#1}}
\newcommand{\CosFrac}[3]{\cos\Parenthesis{ \frac{#1}{#3} #2}}
\newcommand{\Tan}[1]{\tan\Parenthesis{#1}}
\newcommand{\TanFrac}[3]{\tan\Parenthesis{ \frac{#1}{#3} #2}}
\newcommand{\Ln}[1]{\ln\B|\E|}
\newcommand{\LnFrac}[2]{\ln\B|\frac{#1}{#2}\E|}

%Figures Macro  
%\pic{scale}{figureName}{caption}{label}  
\newcommand{\pic}[4]{
\begin{figure}[H]
\centering
\includegraphics[scale=#1]{#2}
\caption{#3} \label{#4}
\end{figure}		}

%\Dpic{figure1}{caption1}{label1}{figure2}{caption2}{label2}
\newcommand{\Dpic}[6]{
    \begin{figure}[H]
  \centering
  \begin{minipage}[b]{0.45\textwidth}
  \label{isometric}
    \includegraphics[width=\textwidth]{#1}
    \caption{#2} \label{#3}
  \end{minipage}
  \hfill
  \begin{minipage}[b]{0.45\textwidth}
  \label{side}
    \includegraphics[width=\textwidth]{#4}
    \caption{#5} \label{#6}
  \end{minipage}
\end{figure}         }

% Gradient operator - Cartesian coordinates 
\newcommand{\gradientCartesian}[1] %\gradientCartesian{v}
{ \PartialFrac{#1}{x}\hat{x} + \PartialFrac{#1}{y}\hat{y} + \PartialFrac{#1}{z}\hat{z} }

% Divergence operator - Cartesian coordinates 
\newcommand{\divergenceCartesian}[3] %\diverganceCartesian{u}{v}{w} 
{ \PartialFrac{#1}{x} + \PartialFrac{#2}{y} + \PartialFrac{#3}{z} }

% Laplace operator - Cartesian coordinates 
\newcommand{\laplaceCartesian}[1] %\laplaceCartesian{u}
{ \PartialFracN{#1}{x}{2} + \PartialFracN{#1}{y}{2} + \PartialFracN{#1}{z}{2} }

% Gradient operator - Cylindrical coordinates  
\newcommand{\gradientCylindrical}[1] %\gradientCylindrical{f} 
{ \PartialFrac{#1}{r}\hat{r} + \frac{1}{r}\PartialFrac{#1}{\theta}\hat{\theta} + \PartialFrac{#1}{z}\hat{z} }

% Divergence operator - Cylindrical 
\newcommand{\divergenceCylindrical}[3] %\divergenceCylindrical{r}{\theta}{a} 
{ \frac{1}{r}\PartialFrac{\Parenthesis{r#1}}{r} + \frac{1}{r}\PartialFrac{#2}{\theta} + \PartialFrac{#3}{z} }

% Laplace operator - Cylindrical coordinates 
\newcommand{\laplaceCylindrical}[1] %\laplaceCylindrical{V} 
{ \frac{1}{r}\PartialFrac{ }{r}\Parenthesis{r\PartialFrac{#1}{r}} + \frac{1}{r^2}\PartialFracN{#1}{\theta}{2} + \PartialFracN{#1}{z}{2} }

% Gradient operator - Spherical coordinates 
\newcommand{\gradientSpherical}[1] %\gradientSpherical{V}  
{ \PartialFrac{#1}{r}\hat{r} + \frac{1}{r}\PartialFrac{#1}{\theta}\hat{\theta} + \frac{1}{r\sin\Parenthesis{\theta}}\PartialFrac{#1}{\phi}\hat{\phi} }

% Divergence operator - Spherical coordinates 
\newcommand{\divergenceSpherical}[3] %\divergenceSpherical{r}{\theta}{\phi} 
{ \frac{1}{r^2}\PartialFrac{\Parenthesis{r^2#1}}{r} + \frac{1}{r\sin\Parenthesis{\theta}}\PartialFrac{}{\theta}\SqBracket{#2\sin\Parenthesis{\theta}} + \frac{1}{r\sin\Parenthesis{\theta}}\PartialFrac{#3}{\phi} }

% Laplace operator - Spherical coordinates 
\newcommand{\laplaceSpherical}[1] %laplaceSpherical{f} 
{ \frac{1}{r^2}\PartialFrac{}{r}\Parenthesis{r^2\PartialFrac{#1}{r}} + \frac{1}{r^2\sin\Parenthesis{\theta}}\PartialFrac{}{\theta}\Parenthesis{\sin\Parenthesis{\theta}\PartialFrac{#1}{\theta}} + \frac{1}{r^2\sin\Parenthesis{\theta}}\PartialFracN{#1}{\phi}{2} }

% Dot Product - Einstein's notation 
\newcommand{\dotEins}[4] %\dotEins{A}{i}{B}{j} 
{ \Sub{#1}{#2}\Sub{#3}{#4}\Sub{\delta}{#2#4} }

% Cross Product - Einstein's notation 
\newcommand{\crossEins}[5] %\crossEins{ijk}{A}{i}{B}{k}  
{ \Sub{\varepsilon}{#1}\Sub{#2}{#3}\Sub{#4}{#5} }

% Stokes Theorem 
\newcommand{\stokesTheorem}[1] %\stokesTheorem{F} 
{ \Sub{\oint}{C}{#1}\cdot d\vec{r} = \iint\Parenthesis{\nabla\times {#1} }\cdot\,d\vec{S} }

% Divergence Theorem 
\newcommand{\divergenceTheorem}[1] %\divergenceTheorem{F}
{ \Sub{\iiint}{\forall}\Parenthesis{\nabla\cdot #1}d\forall = \Sub{\oiint}{S}\Parenthesis{#1\cdot\hat{n}}dS } 

%  Leibniz Integral Rule
\newcommand{\leibnizIntegral} 
{ \frac{d}{dx}\int^{b}_{a}\Function{f}{x,t}dt = \SqBracket{ \Function{f}{x,b}\PartialFrac{b}{x} - \Function{f}{x,a}\PartialFrac{a}{x} } + \int^{b}_{a}\PartialFrac{ }{x} \Function{f}{x,t}dt }  

% Taylor Series 
\newcommand{\taylorSeries} 
{ \sum_{n=1}^{\infty}\frac{f^{(n)}(a)}{n!}\B(x-a\E)^n }

% Material Derivative Eq. - Vector notation 
\newcommand{\materialDerivativeVec}[1] %\materialDerivativeVec{\rho V}
{   \frac{D}{Dt}\B(#1\E) = \frac{\partial}{\partial t}\B(#1\E) + \B(\vec{V}\cdot\nabla\E) #1   }

% Reynolds Transport Theorem 
\newcommand{\reynoldsTransportTheorem}[1] %\reynoldsTransportTheorem{\vec{F}} 
{ \frac{D}{Dt}\int_{C\forall}#1d\forall = \int_{C\forall}\PartialFrac{#1}{t}d\forall + \oint_{CS} \Parenthesis{ \Sub{\vec{V}}{relative} \cdot \hat{n} }#1dS }

% Continuity Eq. 
\newcommand{\continuityVec}
{   \dot{\rho} + \nabla\cdot\Parenthesis{ \rho\vec{V} }   }

% Continuity Eq. - Cartesian coordinates  
\newcommand{\continuityCartesian} %continuityEqExpanded (former name)
{ \dot{\rho} + \PartialFrac{\Parenthesis{\rho u}}{x} + \PartialFrac{\Parenthesis{ \rho v}}{y} + \PartialFrac{\Parenthesis{\rho w}}{z} }

% Continuity Eq. - Cylindrical coordinates 
\newcommand{\continuityCylindrical} 
{ \dot{\rho} + \frac{1}{r}\PartialFrac{ }{r}\Parenthesis{ \rho r\Sub{v}{r} } + \frac{1}{r}\PartialFrac{ }{\theta}\Parenthesis{\rho\Sub{v}{\theta}} + \PartialFrac{ }{z}\Parenthesis{\rho\Sub{v}{z}} }

% Continuity Eq. - Spherical coordinates 
\newcommand{\continuitySpherical} 
{ \dot{\rho} + \frac{1}{r^2}\PartialFrac{}{r}\Parenthesis{ \rho r^2\Sub{v}{r} } + \frac{1}{ r\sin(\theta) }\PartialFrac{}{\theta}\Parenthesis{ \rho\sin(\theta)\Sub{v}{\theta} } + \frac{1}{ r\sin(\theta) }\PartialFrac{}{\phi}\Parenthesis{ \rho\Sub{v}{\phi} } }

% Incompressible Navier Stokes Eq - Material derivative 
\newcommand{\incompressibleNavierStokesMaterial} 
{ \rho\frac{D\vec{V}}{Dt} = -\nabla P + \mu \Up{\nabla}{2}V + \Sub{B}{f} }

% Incompressible Navier Stokes Eq. - Vector notation  
\newcommand{\incompressibleNavierStokesVec} 
{   \rho\Parenthesis{ \dot{V} + \Parenthesis{\vec{V}\cdot\nabla}V } = -\nabla P + \mu \Up{\nabla}{2}V + \Sub{B}{f}  }

% Incompressible Navier Stokes Eq. - Vector expanded
\newcommand{\incompressibleNavierStokesCartesian}[2] %\incompressibleNavierStokesExpanded %\incompressibleNavierStokesCartesian{u}{x}
{ \rho\Parenthesis{ \dot{#1} + u\PartialFrac{#1}{x} + v\PartialFrac{#1}{y} + w\PartialFrac{#1}{z} } = -\PartialFrac{P}{#2} + \mu\Parenthesis{ \PartialFracN{#1}{x}{2} + \PartialFracN{#1}{y}{2} + \PartialFracN{#1}{z}{2} } + \Sub{f}{#2} }

% Incompressible Navier Stokes Eq. - Cylindrical on r direction 
\newcommand{\incompressibleNavierStokesCylindricalR} 
{ \rho\Parenthesis{ \Sub{\dot{v}}{r} + \Sub{v}{r}\PartialFrac{\Sub{v}{r}}{r} + \frac{\Sub{v}{\theta}}{r}\PartialFrac{\Sub{v}{r}}{\theta} -\frac{\Sub{v}{\theta}^2 }{r} + \Sub{v}{z}\PartialFrac{\Sub{v}{r}}{z}}  =   - \PartialFrac{P}{r}  + \mu\Parenthesis{ \frac{1}{r}\PartialFrac{}{r}\Parenthesis{r\PartialFrac{\Sub{v}{r}}{r} } -\frac{\Sub{v}{r}}{r^2}  + \frac{1}{r^2}\PartialFracN{\Sub{v}{r}}{\theta}{2} - \frac{2}{r^2}\PartialFrac{\Sub{v}{r}}{\theta} +  \PartialFracN{\Sub{v}{r}}{z}{2}  } + \Sub{f}{r} }

% Incompressible Navier Stokes Eq. - Cylindrical on theta direction 
\newcommand{\incompressibleNavierStokesCylindricalTheta} 
{ \rho\Parenthesis{ \Sub{\dot{v}}{\theta} + \Sub{v}{r}\PartialFrac{\Sub{v}{\theta}}{r} + \frac{\Sub{v}{\theta}}{r}\PartialFrac{\Sub{v}{\theta}}{\theta} + \frac{\Sub{v}{r}\Sub{v}{\theta}}{r} + \Sub{v}{z}\PartialFrac{\Sub{v}{\theta}}{z}}  = - \frac{1}{r}\PartialFrac{P}{\theta} + \mu\Parenthesis{  \frac{1}{r}\PartialFrac{}{r}\Parenthesis{r\PartialFrac{\Sub{v}{\theta}}{r} } - \frac{\Sub{v}{\theta}}{r^2}  +  \frac{1}{r^2}\PartialFracN{\Sub{v}{\theta}}{\theta}{2} + \frac{2}{r^2}\PartialFrac{\Sub{v}{r}}{\theta} + \PartialFracN{\Sub{v}{\theta}}{z}{2}  } + \Sub{f}{\theta} } 

% Incompressible Navier Stokes Eq. - Cylindrical on z direction 
\newcommand{\incompressibleNavierStokesCylindricalZ} 
{  \rho\Parenthesis{ \Sub{\dot{v}}{z} + \Sub{v}{r}\PartialFrac{\Sub{v}{z}}{r} + \frac{\Sub{v}{\theta}}{r}\PartialFrac{\Sub{v}{z}}{\theta} + \Sub{v}{z}\PartialFrac{\Sub{v}{z}}{z} } = -\PartialFrac{P}{z} + \mu\Parenthesis{ \frac{1}{r}\PartialFrac{ }{r}\Parenthesis{r\PartialFrac{\Sub{v}{z}}{r}} + \frac{1}{r^2}\PartialFracN{\Sub{v}{z}}{\theta}{2} + \PartialFracN{\Sub{v}{z}}{z}{2} }+\Sub{f}{z}  } 

% Incompressible Navier Stokes Eq. - Einstein's notation  
\newcommand{\incompressibleNavierStokesEins}
{   \rho\Parenthesis{ \Sub{\dot{V}}{k} + \Sub{V}{i}\Partial{j}\Sub{V}{k}\Sub{\delta}{ij} } = -\Partial{k}P + \mu\UpSub{\partial}{j}{2}\Sub{V}{k} + \Sub{f}{k}   }

% Non-dimensional Navier Stokes Eq. 
\newcommand{\nonDimensionalNavierStokes} 
{\SqBracket{St}\frac{D\vec{V}^*}{Dt^*} = -\SqBracket{Eu}\nabla^*P^*  +\SqBracket{\frac{1}{Re}}\nabla^{*^{2}}V^* + \SqBracket{\frac{1}{Fr^2}}g^* }

% Shear Strain Tensor 
\newcommand{\shearStrainTensor}
{ \Sub{\varepsilon}{ij} = \frac{1}{2}\Parenthesis{ \PartialFrac{\Sub{v}{i}}{\Sub{x}{j}} + \PartialFrac{\Sub{v}{j}}{\Sub{x}{i}} } }

% Shear Stress Tensor, Stokes Hypothesis is assumed  
\newcommand{\shearStressTensor} 
{ \Sub{\tau}{ij} = 2\mu\SqBracket{ \Sub{\varepsilon}{ij} -\frac{1}{3} \Parenthesis{ \PartialFrac{\Sub{v}{k}}{\Sub{x}{k}}  }\Sub{\delta}{ij} }} 

% Total Stress Tensor (Cauchy Momentum Tensor) 
\newcommand{\cauchyMomentumTensor} %totalStressTensor 
{ \Sub{\sigma}{ij} = -p\Sub{\delta}{ij} + \Sub{\tau}{ij} }

% Maxwell Equations - Differential notation 
\newcommand{\maxwellEqA}
{   \nabla\cdot\bm{E}  = \frac{\rho}{\Sub{\varepsilon}{0}}  } 
\newcommand{\maxwellEqB} 
{   \nabla\cdot\bm{B} = 0   } 
\newcommand{\maxwellEqC} 
{   \nabla\times\bm{E} = - \frac{\partial\bm{B}}{\partial t}    }
\newcommand{\maxwellEqD} 
{   \nabla\times\bm{B} = \Sub{\mu}{0}\bm{j} + \frac{1}{\Up{c}{2}}\frac{\partial\bm{E}}{\partial t}  }

% Quadratic equation 
\newcommand{\quadraticEq}[3] %\quadratic{a}{b}{c} 
{ \frac{  -(#2) \pm \sqrt{ (#2)^2 - 4(#1) (#3) } }{2(#1)} }

% Binomial coeficient 
\newcommand{\binomialCoeff} 
{ \binom{n}{k} = \frac{n!}{k!\Parenthesis{n-k}!} } 

\usepackage[backend=biber, sorting=none]{biblatex}
\addbibresource{../chanl.bib}
\Header{\today}{Literature Review}{Martin E. Liza}
\usepackage{multirow}


\begin{document}
% Weeks 1-2 
\iftrue 
    \section{Physics and Computation of Aero-Optics \cite{wang:arofm:2012}}
        \begin{itemize} 
            \item The timescale for optical propagation is negligibly short relative to flow timescales, and hence optical popagation can be solved in a frozen flow field at each time instant. 
            \item \textbf{OPL:} Optical Path Length. 
            \item \textbf{OPD:} Optical Path Difference.
            \item \textbf{SR:} Strehl Ratio.
            \item $ \Function{OPD}{x,y,t} = \underbrace{ \Function{OPD}{x,y} }_{\text{steady-lensing}} + \underbrace{\SqBracket{ \Function{A}{t}x + \Function{B}{t}y }}_{\text{beam jitter}} + \underbrace{ \Function{OPD}{x,y,t} }_{\text{beam shaping}} $
            \item \textbf{steady-lensing:} function only of the time-averaged density field and impose a steady distortion such as defocus. 
            \item \textbf{beam jitter:} does not change spatial distribution from the outgoing beam but simply redirects it in directions defined by $\Function{A}{t}$ and $\Function{B}{t}$. 
            \item \textbf{beam shaping:} causes the beam to change its shape and intensity distribution. 
            \item Given the OPD profile after the beam passes the turbulence region, one can solve its free space propagation using Fourier optics to obtain the exact far-field projection.
            \item \textbf{SR:} measure of beam quality relative to a diffraction-limited beam at each instant. 
            \item Computation of aero-optics consist of two essential parts: solutions of the aberrating flow field via CFD and propagation of the optical beam through the aberrating flow to the target. 
            \item Beam propagation is computed by a combination of ray tracing with Fourier optics.
            \item To accurately compute the index of refraction field, one must capture turbulence scales over all optically relevant wave numbers and frequencies. 
            \item Euler equations do not describe the correct physics of refractive index fluctuation in a turbulent flow and must rely on numerical dissipation to mimic the effect of physical viscosity. 
            \item Steady RANS calculates all turbulent scales, resulting in an ensemble-average (time-averaged) density field from which the steady-lensing effects can be evaluated but not the beam jitter and beam shaping.      
            \item A hybrid RANS/LES model reduces computational costs for wall-bounded flows at high Reynolds numbers.
            \item A lack of flow resolution will cause errors in the computation of the optical phase when the beam is traced through the turbulence flow.
            \item Optical effects are negligible at a small scale in turbulence flow. 
            \item Experiments suggested that the majority of optically active structures in compressible boundary layers are located in the outer region of the boundary layer, moving at $0.82 - 0.85$ times the free stream velocity.  
            \item The predominant mechanism for density fluctuations in the compressible boundary layers is thought, to be adiabatic heating/cooling due to velocity fluctuations via the strong Reynolds analogy, as pressure fluctuations inside the boundary layers are much smaller than temperature fluctuations. 
            \item Optical distortions by the fully developed wake were insensitive to the Reynolds number, whereas distortions by the separated shear layers were sensitive to it.
        \end{itemize}


    \section{Aero-optical effects in non-equilibrium air \cite{tropina:plc:2018}}
        \begin{itemize}
            \item In a flow environment, which may induce real gas effects and chemistry, the refractive index depends on the air density, composition and internal state. 
            \item In many cases, flows behind shocks and in boundary layers are not in thermal equilibrium and thus may have associate with them thermal aero-optical effects significantly different from those associated with thermal equilibrium. 
            \item The refractive index of air is proportional to the gas density, with the constant of proportionality known as the Gladstone-Dale constant. $\Sub{R}{g} = \frac{\Function{\alpha}{\omega}}{2\Sub{\epsilon}{0}}$, where: $\Function{\alpha}{\omega}$ is the polarization constant and $\Sub{\epsilon}{0}$ is the permittivity in vacuum. 
            \item The Gladstone-Dale constant might not be constant at high speed flows because particles are tend to be found in their excited state. 
            \item $n = 1 + \Sub{R}{g}N$, where: $n$ is the index of refraction and $N$ is the density of the gas in molecules per cubic meter.  
            \item In thermal equilibrium the fractional population and polarization obeys the Boltzmann distribution; however, for a non-equilibrium, the fractional population and polarization does not necessary obeys the Boltzmann distribution.  
            \item The total energy of a state is given by the summation of electronic, vibrational and rotational energies; which are determined by the molecular vibrational and rotational constants.
            \item A gas with a higher temperature will exhibit a larger refractive index. 
            \item For nitrogen and oxygen, vibrational excitation has a significantly larger effect on the polarization compared with the rotational excitation. 
        \end{itemize}


    \section{Aero-Optic Distortion in Transonic and Hypersonic Turbulent Boundary Layers \cite{wyckhman:aiaa:2009}}
        \begin{itemize}
            \item In a compressible boundary layer, the density, temperature, pressure and velocity vary in space and time. However, only the density fluctuations are relevant to optical distortion. 
            \item The density fluctuations can be related to the velocity fluctuations using the Strong Reynolds Analogy (SRA). 
            \item Accepting that the pressure fluctuations are small, the ideal gas law for small fluctuations leads to: $\frac{\rho^{\prime}}{\bar{\rho}} = \Parenthesis{\gamma -1} M^2\frac{u^{\prime}}{\bar{U}} $
            \item From previous equation, it is expected that aero-optical distortions increase at higher Mach numbers and lower altitudes. 
            \item The linking equation assumes that the density fluctuations are random and normally distributed an that the integration length $L$ is much greater than the integral scale $\Lambda$.
            \item $\UpSub{OPD}{rms}{2} = 2 \UpSub{R}{g}{2} \int^{L}_{0} \bar{\Function{\rho^{{\prime}^2}}{y}} \Function{\Lambda}{y}dy  $ Linking equation. 
            \item In the inner region of a turbulent boundary layer, the structures tend to organize into long, streamwise streaks; the inner region extends to approximately $10-15\%$ of the boundary layer thickness.
            \item In the outer layer, of the boundary layer is dominated by structures sometimes termed large scale motions. 
            \item Because the phase distortion is integrated through the boundary layer, the contributions from the larger scale structures in the outer region wold be expected to be significant. However, the linking equation shows that the variation of relative density fluctuations with height above the wall must also be taken into consideration. The density fluctuations are stronger in the inner half of the boundary layer, but in a compressible flow the local density decreases sharply close to the wall especially in high Mach number flows. 
            \item There are two different ways used to describe the degree of aberration in a wave front: the root-mean-square variation across the wave front, and the SR, defined as the ratio between the measure of irradiance in the far field to the diffraction limited theoretical maximum. 
            \item Two ways of calculating SR. SR is calculated at a fixed point (centerline); others calculate the SR at the point of maximum irradiance, regardless of its location. 
        \end{itemize}


    \section{HyperCode:A framework for high-order accurate turbulent non-equilibrium hypersonic flow simulation \cite{vogiatzis:scitech:2020}} 
        \begin{itemize}
            \item Coupled, multi-scale, multi-physics interactions not only have a significant role in signal propagation and sensing in extreme environments but also affect crucial decisions regarding thermal management, path planning and control.
            \item \textbf{HyperCode:} code that has unique non-equilibrium prediction capabilities for multicomponent hypersonic flows.
            \item \textbf{OTF:} Optical Transfer Function.  
            \item \textbf{PSF:} Point Spread Function.
            \item For temporal behavior one should first integrate OTFs or PSFs over an exposure window before calculating spatial parameters. This procedure yields primarily information for image or beam quality, since wave-ront and phase errors are readily available.
        \end{itemize}
        

    \section{Dome and mirror seeing estimates from the Thirty Meter Telescope \cite{pazder:spie:2006}}
        \begin{itemize}
            \item \textbf{Mirror Seeing:} degradation of the telescope image due to variations in the index of refraction of the air. 
            \item \textbf{Dome Seeing:} degradation of telescope image quality due to variations in the index of refraction of the air by the dome. 
            \item \textbf{PSSn:} Point Source Sensitivity normalized. This metric is representative of the relative integration time of a background limited point source observation normalized to the atmospheric seeing.
            \item Its assumed that the diffraction image is computed by summing the OPD error through the system, applying this to the exit pupil, and propagating from the exit pupil using the Fourier transform of the exit pupil ignoring the propagation effects within the system. 
            \item Near field diffraction effects are ignored due to the large optical paths involved in the telescope and the large thermal mass of the mirror.
            \item Matlab script was used to perform calculations.
        \end{itemize}
    

    \section{Role of High Fidelity Nonequilibrium Modeling in Laminar and Turbulent Flows for High Speed ISR Missions \cite{vogiatzis:tp:2016}}
        \begin{itemize}
            \item \textbf{DNS:} Direct Numerical Simulations.
            \item DNS of turbulence are performed for low Mach number compressible channel ideal gas flow. 
            \item \textbf{STS:} State to State kinetics. 
            \item \textbf{LLTR:} Lumped Landau Teller Relaxation.
            \item Laminar OPD's exhibit a tilt aberration with a positive gradient in the flow direction while in case of turbulent flows the integral length scale of the flow begins to dominate the OPD spatial behavior resulting in higher order aberrations.
            \item The complexity of hypersonic turbulent nonequilibrium boundary layers still merits the development of models that will retain both the spatial and temporal aero-optical behavior, that can lead to incorporating aberration mitigation strategies into sensor design.  
            \item Hypersonic flow solvers currently employ simple algebraic or two equation turbulence models coupled with thermochemical nonequilibrium models which involve when needed macroscopic approximations and may therefore not be suitable or ISR studies due to the simplifications that prevent the proper resolution of the space and time scales that may affect the signal. 
            \item Many current off the shelf (COTS) software technologies in this context lack key mechanisms that describe the interplay between turbulence and nonequilibrium distributions of molecules over vibrational, rotational and/or translational states. This leads to high uncertainty levels in the results and thus to potentially significant errors in signals analysis; other affected predictions included thrust, drag, aerodynamic heating, surface temperatures, chemical compositions, boundary layer transition and ablation.
            \item Nonequilibrium effect due to Internal energy exchanges between modes of vibration, rotation, translation, dissociation, recombination and radiation are calculated by two types of models.
            \item The LLTR model allows to calculate the coupling of vibrational-translational and dissociation reaction mechanisms on the flow physics, under the assumption that the molecule behaves as a harmonic oscillator. 
            \item The STS model allows for prediction of nonequilibrium molecular distributions of energy among various internal energy modes, as well as state specific transport coefficient.
            \item Detailed representation of molecular distributions over internal energy states may be necessary for proper predictions of surface heat transfer and signal propagation errors.
            \item In conducting an aero-optics analysis o the signal propagation in high speed ISR missions, one would have to study teh propagation of E\&M waves through a nonequilibrium flow field.
            \item The refractive index $N$ of air is given by: $N-1 = \rho\Parenthesis{A+\frac{B}{\lambda^2}}$, where $\rho$ is the mixture density, $\lambda$ is the signal's wavelength, and $A$ and $B$ are constants depending on the air species. 
            \item The OPL is the distance light travels in vacuum in the same time it travels a distance $L$ in a medium. 
            \item The difference between two OPL's at two locations is called OPD.
            \item An OPD map, also called wave-front, is the two-dimensional spatial distribution over an aperture of the deviation of the OPL of the signal from a reference distance.
            \item The length scale over which the phase error due to passage through a given column of medium equals to $1\Units{radian}$ is called the coherence length $\Sub{r}{0}$; which is directly tied to the turbulence correlation and integral length scales. 
            \item The coherence time is defined by the ratio between the coherence length and the average medium velocity relative to the sensor window; it sets the lower limit for exposure time before serious image degradation. 
            \item An optical mesh of resolution comparable to that of the computational mesh and aligned with the signal propagation direction is defined. 
            \item For a single flow field, either a steady state simulation converged solution or at a corresponding time of an unsteady solution, the process can be as follow: calculate the OPL for each cell o the optical mesh, then integrate the paths to obtain the total OPL for each point on the aperture. Subtracting the reference OPL, yields the OPD map. The OPD map can be further post processed for various optical parameters; such as SR, OTF, PSF.   
        \end{itemize}


	\section{Aero-thermal simulations of the TMT Laser Guide Star Facility \cite{vogiatzis:spie:2014}} 
        \begin{itemize}
            \item The time-step used in the initial simulation was $300\Units{s}$. This is too coarse for beam jitter estimates but it is sufficient for temperature calculations.
            \item The grid of each region has been generated in such a way that each interface facet is common to the two cells of the two region involved. Convection, conduction and radiation heat transfer between regions is taking place through all interfaces.
        \end{itemize}
		

    \section{Analytical Approach for Aero-Optical \& Atmospheric Effects in Supersonic Flow Fields \cite{gupta:scitech:2020}} 
        \begin{itemize}
            \item The refractive index of a medium, such as air, governs the angular shift in the path of an optical signal. For a fluid the refractive index is a function of the thermodynamic state. 
            \item The mix of dissociation and recombination reactions at high-supersonic and hypersonic speeds is the cause of communication blackout.
            \item At these speeds radio waves are unable to penetrate through the charged plasma's layer. Hence, RF communication is no longer effective.
            \item When an optical signal passes through a turbulent flow field, it undergoes deviation (due to refractive index changes) and distortion (due to refractive index fluctuations).
            \item Density ratio in supersonic flow increases with increasing Mach number and so would the downstream density. 
            \item When a signal travels through a shock wave, the signal will deviate (due do the density gradient from the upstream and downstream regions in the flow field) at each intermediate interface of the shock layer.  
            \item A shock wave in a supersonic flow can be consider as an interface between two medias of different refractive indices (think about Snell's law derivation). 
            \item When refraction angle is the same as the shock angle, the signal propagates straight down. 
            \item Shock layer thickness is relative small at high Mach numbers as the shock waves is closer to the body. 
        \end{itemize}
		

    \section{Optical distortion caused by propagation through turbulent shear layer \cite{pade:spie:2003}} 
        \begin{itemize} 
            \item The fluctuations of the index of refraction are calculated using Wye's model.
            \item Methods use to investigate aero-optics problems: (1) Measuring the deviation of the optical radiation while passing through a flow field. (2) Finding an expression for the variations in the index of refraction (calculating fluid density). 
            \item It is well known that the relation between the index of refraction and the density is given by the Gladstone Dale formula, which is valid for moderate high Mach number airflow.   
            \item A shear narrow shear layer perpendicular to the laser beam's path, with a width of $0.03\%$ of the path length causes a strong reduction in resolution.
            \item The reduction in resolution is dependent on the distance of the shear layer from the exit aperture of the beam. 
            \item The turbulent limited resolution is inversely proportional to the Fried's coherence length.
        \end{itemize}


    \section{Modeling Radio Communication Blackout and Blackout Mitigation in Hypersonic Vehicles \cite{kundrapu:jsr:2015}} 
        \begin{itemize} 
            \item Weakly ionized plasma generated around the surface of a hypersonic reentry vehicle is simulated using full Navier-Stokes equations in multi-species single fluid form.
            \item The electromagnetic wave's interaction with the plasma layer is modeled using multi-fluid equations for fluid transport and full Maxwell's equations for the electromagnetic fields. 
            \item Shock waves convert the kinetic energy to internal energy, an thereby increases the fluid temperature significantly.  
            \item The electrons in the plasma layer may interrupt the propagation of RF waves when the plasma oscillation frequency is higher than the RF frequency. 
            \item Steps: (1) modeling and simulation of the multi-species hypersonic flow over the vehicle to obtain plasma density distribution, (2) validation of the plasma density distribution with the results from literature, (3) modeling of RF waves propagation into the plasma and validation with the dispersion relations, (4) the propagation of a plane E\&M wave on to the vehicle's surface through the plasma layer using a magnetic window and the whistler wave conversion.
            \item The properties' viscosity, thermal conductivity (mole fraction averaging), and specific heat (mass fraction average) of the individual species are obtained from the kinetic theory of gases. Gas constant is computing using the mole fraction averaged molecular weight. 
            \item The interaction between RF waves and plasma are calculated by a MHD model without the assumption of quasi neutrality, without the assumption that the light wave is infinitely fast and by using full Ohm's law (including electron inertia) in the MHD system. 
            \item The plasma frequency timescale is much smaller than the advection, diffusion and collision timescales; ergo these terms can be neglected. 
            \item Wall temperature is lower than the boundary layer temperature. 
            \item Inclusion of radiation looses from the plasma and the diffusion of electrons in the simulation could decrease the density to some extent. 
            \item The dispersion relation is derived from the two fluid electromagnetic plasma model and written in terms of plasma parameters and the speed of light.
            \item The plasma E\&M wave does not propagate below the plasma frequency; this is the reason for radio communication blackout.
        \end{itemize}


    \section{Assessment of Hypersonic Flow Physics on Aero-Optics \cite{mackey:aiaa:2019}} 
        \begin{itemize} 
            \item The computations are performed with the aid of finite rate chemistry and the Landau-Teller vibrational energy relaxation model. 
            \item High kinetic energy of the oncoming flow causes the molecules in the flow to be thermally excited, leading to dissociation. 
            \item To accurately assess signal propagation through hypersonic flow field, high-enthalpy-flow physics need to be appropriately accounted for in optical distortion predictions. 
            \item When flow passes through region of high gradient enthalpy, it takes a finite amount of collisions for the internal energies to adjust to a new thermodynamic state, often resulting in regions of nonequilibrium and relaxation.  
            \item Vibrational and chemical energy exchange processes are often relatively slow compared with translational and rotational energy exchange processes. 
            \item In nonequilibrium regions, chemical dissociation often takes place at freestream velocities greater than Mach $6$. 
            \item High enthalpy flow effects usually manifest as a relatively thin boundary later with steep gradient in density. 
        \end{itemize}


    \section{Influence of vibrational non-equilibrium on the polarizability and refraction index in air: computational study \cite{tropina:jop:2019}}
        \begin{itemize} 
            \item Knowledge of variations of refractive index in air can ensure the reliable operation of optical on-board instrumentation such as optical air-data systems, airborne LIDAR, and directed energy systems.
            \item Polarizability in general, is represented by the polarizability tensor, reflecting a fact that an applied electric field can induce different polarization components. In its turn the scalar polarizability is an average of the diagonal elements of the polarizability tensor. 
            \item Commonly tabulated values of $\Sub{R}{g}$ and $\alpha$ are given only for atoms and molecules in their ground state. 
            \item Atomic species are not often found in appreciable quantities in their excited electronic states in hypersonic flows, it is well-known fact that their polarizabilities can be significantly larger than the ground state.
            \item From the perspective of molecular dynamics theory, polarizability dependence on temperature is caused by centrifugal stretching and anharmonicity of the intramolecular potential. 
            \item Increase of the vibrational temperature leads to an increase of the polarizability due to overpopulation of higher lying vibrational states and, depending on the level of vibrational nonequilibrium and gas temperature, leads to the polarizability increase on the order of $1\%-9.5\%$. 
        \end{itemize}

        
    \section{Table of Gladstone-Dale constants on the ground state}
   		\begin{table}[h]
   		    \centering 
            \begin{tabular}{ | c | c c c c c | c | }
                \hline
                $ \times 10^{-3} \UnitsFrac{m^3}{kg} $ & $\Sub{\text{N}}{2}$  & N & $\Sub{\text{O}}{2}$ & O & NO & Ref. \\ [0.5ex]
                \hline \hline
                \multirow{4}{*}{$\Sub{R}{GD}$} & $2.38$  & $3.01$  & $1.90$ & $1.82$ & $2.21$     & \cite{karl:fdc:2003}     \\
                 & & & $1.93$  & $2.04$ & &  \cite{anderson:tpof:1969}  \\
                 & & $3.10$ & & $1.80$ & & \cite{alpher:tpof:1959} \\
                 & $2.38$ & & $1.89$ & $1.73$ & &  \cite{qin:caf:2002}  \\
                \hline
            \end{tabular}
            \label{tab:species}
        \end{table}


\fi

% Weeks 2-4
\iftrue 
    \section{Atmospheric Propagation Vs. Aero-Optics \cite{siegenthaler:aiaa:2008}} 
        \begin{itemize} 
            \item In the case of atmospheric propagation, the index of refraction variations are due to temperature fluctuations caused by very large scale (relative to the beam aperture) temperature gradients in a region of the atmosphere that itself is n shear and has turbulence scales that cascade from the largest scales down to sub aperture scales.
            \item In the case of aero-optics, the index of refraction variations for weakly-compressible flows are caused by density fluctuations that form in and around the coherent structures in the turbulent flow. 
            \item At the heart of the atmospheric propagation, a single parameter $\Sub{C}{n}^2$. This parameter describes not only the scale size of the aberrating index of refraction fluctuations, but also the magnitude of the aberration with them. 
            \item In aero-optics the $\Sub{C}{n}^2$ parameter is irrelevant, because of the characterization of aero-optic turbulence since the range of scales over which aero-optic turbulence can be consider Kolmogorov places them below the scale sizes that are relevant to the optical problem.  
            \item The Kolmogorov model of turbulence; is a statistical, incompressible model; which based largely on the idea of energy being transferred between various length scales in the flow. At the largest scales, persisting eddies act as the energy supply for the turbulent flow. At the small scales, viscous forces become dominant, and the kinetic energy in the flow is dissipated as heat. 
            \item Kolmogorov proposed that in fully developed turbulence flow, the properties in the inertial range were only dependent upon the flux of energy though those scales. Note, Kolmogorov' turbulence model is assumed to be an incompressible flow.  
            \item Variations in temperature and pressure of a gas or collection of gases will produce variations in density.
            \item In air, for wavelengths in or near the visible range, the index of refraction can be approximated to be: $ dn = 7.77\times 10^{-7} \frac{P}{T}\Parenthesis{\frac{dP}{P} - \frac{dT}{T}} $
            \item In the free atmosphere, pressure gradients severe enough to produce significant density variation on the length scale of most optical path diameters tend to disperse at sonic speeds and the small pressure fluctuations that are coupled to velocity variations in atmospheric turbulence tend to have a low order optical impact compared to the effect of temperature fluctuations.  
            \item As light travels slower in areas with a higher index of refraction, the same absolute path length becomes effectively longer or shorter from an optical standpoint in regions of greater or lesser index of refraction.
            \item The OPD can be scaled by the wavelength of light to indicate differences in phase over the recieving plane. These differences in phase produce differences in amplitude and intensity as a wave propagates.
            \item Fried, assumed that if one was attempting to recover information from light received through an aperture, then there would be some signal modulated onto a carrier wave, such that the amplitude of the carrier would be much greater than the amplitude of the signal. 
            \item Having an aperture larger than the Friend's parameter $\Parenthesis{\Sub{r}{0}}$, will not significantly improve the signal to noise ratio of a signal or resolution of an image. 
            \item The Friend's parameter is physically defined as the size of an aperture over which the root-mean square phase variance is approximately $1\Units{radian}$.
            \item The Greenwood frequency is commonly used as a guideline for the bandwidth required of a corrective system intended to deal with optical turbulence.
            \item The assumptions of isotropy and homogeneity of the flow don't apply in aero-optics.
            \item Experimental studies of transonic flow have shown that resulting flow structures are a source of optical aberrations.  
            \item Studies have shown that in transonic shear layers, there are rolling vortices that make up flows containing significant wells of low pressure. These low pressure regions can produce significant optical distortions not predicted by models based solely on temperature differences.  
            \item Shear layers are of particular interest in aero-optics as they may form over cavities or in any instance of separated flow. 
            \item Aero-optic flows do not follow atmospheric flows' rules of scale, because they tend to have a characteristic length and associated frequency. This is because these flows are not in equilibrium state as is it assumed for Kolmogorov turbulence's model. 
            \item A possible corrective filter with a transfer function that remove optical distortions can be derived from the Greenwood frequency. 
        \end{itemize}

    \section{Study on the Effect of Compressibility and Knudsen number on Aero Optics in Supersonic/Hypersonic Flows \cite{ren:fdc:2012}} 
        \begin{itemize} 
            \item \textbf{Kn:} Knudsen number; is a dimensionless number defines as the ratio of the molecular mean free path length $\Parenthesis{\lambda}$ to a representative physical length $\Parenthesis{L}$. This ratio helps to determine where statistical mechanics or continuum mechanics formulation of fluid dynamics should be used to model a situation. If $Kn$ is $\geq 1$, then $\lambda \approx L$ and the continuum assumption of fluid mechanics is no longer a good approximation; hence, statistical methods should be use. $Kn = \frac{\lambda}{L}$
            \item A smaller Knudsen number would lead to a larger optical distortion because the media was denser and larger strength per unit which will lead to a larger distortion. $\Sub{OPD}{RMS} \approx \Up{Kn}{-1}$.
            \item Compressibility played an important role in optical distortion. For supersonic flows, optical distortion was lower at higher Mach number $\Parenthesis{\Sub{OPD}{RMS}\approx\Up{Ma}{-1}}$; and for hypersonic flows, optical distortion became more severe as Mach number increased $\Parenthesis{\Sub{OPD}{RMS}\approx Ma}$.
            \item It was confirmed that shock wave regions contributed most to the distortion. Near wall flow regions contributed less than $10\%$ to the total distortion in supersonic flows; and in hypersonic flows the near wall flow regions contributed $40\%$ to the total distortion. 
            \item The blurring due to a hypersonic flow, can lead to a decrease in the signal to noise ration and image degradation, consequently affecting the target's location. (Cohesive Reference Point CRP)     
            \item Navier-Stokes equations might not be capable to resolve the physical process at order of magnitudes less than nano-meters$/$seconds and$/$or micro-meters$/$seconds. Thereby, a numerical method for Boltzmann equation becomes necessary to capture the physical phenomenon in the small scale. 
            \item Flow fields were computing using Direct Simulation Monte Carlo (DSMC) and a new ray tracing technique was proposed to computer the phase difference from the index of refraction.  
            \item The aero-optical analysis, a new ray tracing technique based on CFD meshes was proposed, and it was assumed that the grid resolution from the CFD simulation was considered to be an accurate representation of the aero-optical effects. 
            \item OPD is also called the relative phase $\Parenthesis{\phi}$.  
            \item The optically-active region in supersonic flows was larger than in hypersonic flows, because of larger compressibility.  
            \item The aero-optical effects caused by boundary layers in hypersonic flows was more severe than in supersonic flows. 
        \end{itemize}

\fi

\newpage
\printbibliography[title=Bibliography]
\end{document}
